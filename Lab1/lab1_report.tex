\documentclass{article}

%% Page Margins %%
\usepackage{geometry}
\geometry{
    top = 0.75in,
    bottom = 0.75in,
    right = 0.75in,
    left = 0.75in,
}

\usepackage{amsmath}
\usepackage{graphicx}
\usepackage{tikz}
\usepackage{parskip}

\title{Lab 1: Building Circuits}

\author{Wendy Wan}

\begin{document}
\maketitle

The preparation exercises for this lab involve the design of digital circuits using AND, OR and NOT gates that implement a specified behaviour.
However, to make this work in the real world, these gates are built into the 7400-series chips as described in the lab handout.
So your circuit diagrams also need to specify how many of these chips your design needs, and how the pins on the chips you use correspond to the pins on your circuit diagram.
Luckily, Logisim Evolution includes this information.

Your task is to design the circuits in Parts I and II using \textbf{only 74LS04 (NOT), 74LS08 (AND) and 74LS32 (OR) chips}.
In Logisim Evolution, these correspond to 7404, 7408, and 7432, respectively, under \verb|TTL|.
Choose the actual pin numbers of the chips that you will use when you build your circuit and show them on your circuit diagram;
this makes the construction of your physical circuit easier.

For each logic function, show all of the steps required to go from the specification to the final circuit.
This includes:

\begin{itemize}
\item Deriving a truth table
\item Drawing a schematic of the final circuit
\item Assigning names to input and output wires on your schematic
\item Deriving a simplified logic expression, when applicable
\end{itemize}

You can and should use Logisim Evolution to help you.
The \verb|Poke| tool helps you debug your schematic.
The \verb|Analyze Circuit| option helps you compare your own truth table derivation with what Logisim infers from your schematic.
When your schematic is complete, use Logisim's \verb|File -> Export Image| so that you can include it in your report.
You can also export the contents of \verb|Analyze Circuit| into Latex, which includes the truth table.
An example is demonstrated on the next page.
You can use it as a guide for Parts I and II.

\newpage
\section*{Part I}

Consider the Boolean function: $f = x\overline{s}+ys$ (a 2-to-1 multiplexer).
Perform the following steps.

\begin{enumerate}

\item Write out the truth table for the function.

\begin{center}
\begin{tabular}{ccc|c}
$x$&$y$&$s$&$f$\\
\hline
$0$&$0$&$0$&$0$\\
$0$&$0$&$1$&$0$\\
$0$&$1$&$0$&$0$\\
$0$&$1$&$1$&$1$\\
$1$&$0$&$0$&$1$\\
$1$&$0$&$1$&$0$\\
$1$&$1$&$0$&$1$\\
$1$&$1$&$1$&$1$\\
\end{tabular}
\end{center}

\item Draw a schematic that implements the function using only the allowed chips.

\begin{figure}[!ht]
    \centering
    \includegraphics[width=0.7\textwidth]{lab1_part1.png}
    \caption{A schematic of a 2-to-1 multiplexer.}
    \label{f:part1}
\end{figure}

\item Is there a cheaper implementation for your design, assuming you are still limited to using the same three chip types?
    Explain why or why not.
    If there is, include a schematic and discuss how much cheaper it is.

There isn't a cheaper implementation for my design since the existing function already is a straightforward expression of a 2-to-1 multiplier. The control signal is essential to select between the x and y input and any other attempts at simplification may get rid of the structure of the multiplexer.

\end{enumerate}

\newpage
\section*{Part II}

Consider the Boolean function: $f = \overline{a + b} + c\overline{b}$.
Perform the following steps.

\begin{enumerate}

\item Write out the truth table for the function.

\begin{center}
\begin{tabular}{ccc|c}
$a$&$b$&$c$&$f$\\
\hline
$0$&$0$&$0$&$1$\\
$0$&$0$&$1$&$1$\\
$0$&$1$&$0$&$0$\\
$0$&$1$&$1$&$0$\\
$1$&$0$&$0$&$0$\\
$1$&$0$&$1$&$1$\\
$1$&$1$&$0$&$0$\\
$1$&$1$&$1$&$0$\\
\end{tabular}
\end{center}

\item Draw a schematic that implements the function using only the allowed chips.

\begin{figure}[!ht]
    \centering
    \includegraphics[width=0.7\textwidth]{lab1_part2.png}
    \caption{A schematic of the function from Part II.}
    \label{f:part2}
\end{figure}

\item Is there a cheaper implementation for your design, assuming you are still limited to using the same three chip types?
    Explain why or why not.
    If there is, include a schematic and discuss how much cheaper it is.

Yes. We start by minimizing the original equation.
\begin{align*}
    f   &= \overline{a + b} + c\overline{b}\\
        &= \overline{a}\overline{b} + c\overline{b}\\
        &= \overline{a} + c
\end{align*}

The schematic for the minimized equation would require a 74LS04 (NOT) chip and a 74LS32 (OR) chip (Figure~\ref{f:part2_reduced}), which is 1 chip (and 3 gates) less than the original schematic (Figure~\ref{f:part2}).

\begin{figure}[!ht]
    \centering
    \includegraphics[width=0.35\textwidth]{lab1_part2_reduced.png}
    \caption{A schematic of the function, with reduced cost.}
    \label{f:part2_reduced}
\end{figure}

\end{enumerate}

\newpage
\section*{Part III}

Draw a schematic of the 2-to-1 mux, but instead of using the \verb|TTL| chips, use the NOT, AND, and OR gates directly (i.e., under \verb|Gates|).

\begin{figure}[!ht]
    \centering
    \includegraphics[width=0.5\textwidth]{lab1_part3.png}
    \caption{A schematic of the 2-to-1 multiplexer from Part III.}
    \label{f:part3}
\end{figure}

\end{document}